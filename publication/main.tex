% This is samplepaper.tex, a sample chapter demonstrating the
% LLNCS macro package for Springer Computer Science proceedings;
% Version 2.21 of 2022/01/12
%
\documentclass[runningheads]{llncs}
%
\usepackage[T1]{fontenc}
% T1 fonts will be used to generate the final print and online PDFs,
% so please use T1 fonts in your manuscript whenever possible.
% Other font encondings may result in incorrect characters.
%
\usepackage{graphicx}
% Used for displaying a sample figure. If possible, figure files should
% be included in EPS format.
%
% If you use the hyperref package, please uncomment the following two lines
% to display URLs in blue roman font according to Springer's eBook style:
%\usepackage{color}
%\renewcommand\UrlFont{\color{blue}\rmfamily}
%\urlstyle{rm}
%
\begin{document}
%
\title{Masked Computation of the Floor Function and Its Application to the Falcon Signature}
%
\titlerunning{Masked Floor Function For Falcon}
% If the paper title is too long for the running head, you can set
% an abbreviated paper title here
%
\author{Pierre-Augustin Berthet\inst{1,3}\orcidID{0009-0005-5065-2730} \and
Justine Paillet\inst{2,3}\orcidID{0009-0009-6056-7766} \and
C\'edric Tavernier\inst{3}\orcidID{0009-0007-5224-492X}}
%
\authorrunning{P-A Berthet et al.}
% First names are abbreviated in the running head.
% If there are more than two authors, 'et al.' is used.
%
\institute{
  Télécom Paris, Palaiseau, France, \email{berthet@telecom-paris.fr}
  \and
  Université Jean-Monnet, Saint-\'Etienne, France, \email{}
  \and
  Hensoldt SAS FRANCE, Plaisir, France, \email{pierre-augustin.berthet,cedric.tavernier@hensoldt.net}
}
%
\maketitle              % typeset the header of the contribution
%
\begin{abstract}
With the ongoing standardization of new POst-Quantum Cryptography (PQC) primitives by the National Institute of Standards and Technology (NIST), it is important to investigate the robustness of new designs to Side Channel Analysis (SCA). Amongst those future standards is Falcon, a lattice-based signature which relies of rational numbers. It thus requires an implementation using floating point arithmetic, which is harder to design well and secure. While recent work proposed a solution to mask the addition and the multiplication, some roadblocks remains, most noticeably how to protect the floor function. In this work we propose \textbf{SEVERAL?A} masking countermeasure to protect the computation of the floor function. We provide mathematical proofs of our method as well as formal security proof. We also discuss its application to the Falcon Signature.

\keywords{Floor Function \and Floating-Point Arithmetic \and Post-Quantum Cryptography \and FALCON \and Side-Channel Analysis \and Masking}
\end{abstract}
%
%
%
\section{Introduction}
With the rise of quantum computing, mathematical problems which were hard to solve with current technologies will be easier to breach. Amongst the concerned problem is the Discrete Logarithm Problem (DLP) which can be solved in polynomial times by the Shor quantum algorithm \cite{shor}. As much of the current asymmetric primitives rely on this problem and will be breach, new cryptographic primitves are studied. The National Institute of Standards and Technology (NIST) launched a post-quantum standardization process \cite{chen2016}. The finalists are CRYSTALS-Kyber \cite{kyber,fips203}, CRYSTALS-Dilithium \cite{dilithium,fips204}, SPHNICS+ \cite{sphincs,fips205} and FALCON \cite{falcon}.

\medskip

\noindent Another concern for the security of cryptographic primitives is their robustness to a Side-Channel opponent. Side-Channel Analysis (SCA) was first introduced by Paul Kocher \cite{kocher} in the mid-1990. This new branch of cryptanalysis focuses on studying the impact of a cryptosystem on its surroundings. AS computations take time and energy, an opponent able of accessing the variation of one or both could find correlations between its physical observations and the data manipulated, thus resulting in a leakage and a security breach. Thus, the study of weaknesses in implementations of new primitives and the ways to protect them is an active field of research.

\medskip While they have been many works focusing on CRYSTALS-Dilithium and CRYSTALS-Kyber, summed up by Ravi et al. \cite{ravi2024}, FALCON is noticeably harder to protect. Indeed, the algorithm relies on floating-point arithmetic, for which there is little litterature of how to protect it.

\subsubsection{Related Work} Previous works have identified two main weaknesses within the signing process of Falcon : the pre-image computation and the Gaussian sampler. The pre-image computation was proved vulnerable by Karabulut and Aysu \cite{Karabulut} using an ElectroMagnetic (EM) attack. Their work was later improved by Guerreau et al. \cite{Guerreau}. To counter those attacks, Chen and Chen \cite{ChenChen} propose a masked implementation of the addition and multiplication of FALCON. However, they did not delved into the second weakness of Falcon, the Gaussian sampler.\newline
The Gaussian sampler is vulnerableto timing attacks, as shown by previous work \cite{allworksseeChenChen}. A isochronous design was proposed by Howe et al. \cite{Howe}. However, a successful single power analysis (SPA) was proposed by Guerreau et al. \cite{Guerreau} and further improved by Zhang et al. \cite{zhang}. There is currently no masking countermeasure for FALCON's Gaussian Sampler. Existing work \cite{mitaka} tends to re-write the Gaussian SAmpler to remove the use of floating arithmetic. 

\subsubsection{Our Contribution}

\subsection{}
\section{Introduction}
\subsection{A Subsection Sample}
Please note that the first paragraph of a section or subsection is
not indented. The first paragraph that follows a table, figure,
equation etc. does not need an indent, either.

Subsequent paragraphs, however, are indented.

\subsubsection{Sample Heading (Third Level)} Only two levels of
headings should be numbered. Lower level headings remain unnumbered;
they are formatted as run-in headings.

\paragraph{Sample Heading (Fourth Level)}
The contribution should contain no more than four levels of
headings. Table~\ref{tab1} gives a summary of all heading levels.

\begin{table}
\caption{Table captions should be placed above the
tables.}\label{tab1}
\begin{tabular}{|l|l|l|}
\hline
Heading level &  Example & Font size and style\\
\hline
Title (centered) &  {\Large\bfseries Lecture Notes} & 14 point, bold\\
1st-level heading &  {\large\bfseries 1 Introduction} & 12 point, bold\\
2nd-level heading & {\bfseries 2.1 Printing Area} & 10 point, bold\\
3rd-level heading & {\bfseries Run-in Heading in Bold.} Text follows & 10 point, bold\\
4th-level heading & {\itshape Lowest Level Heading.} Text follows & 10 point, italic\\
\hline
\end{tabular}
\end{table}


\noindent Displayed equations are centered and set on a separate
line.
\begin{equation}
x + y = z
\end{equation}
Please try to avoid rasterized images for line-art diagrams and
schemas. Whenever possible, use vector graphics instead (see
Fig.~\ref{fig1}).

\begin{figure}
\includegraphics[width=\textwidth]{fig1.eps}
\caption{A figure caption is always placed below the illustration.
Please note that short captions are centered, while long ones are
justified by the macro package automatically.} \label{fig1}
\end{figure}

\begin{theorem}
This is a sample theorem. The run-in heading is set in bold, while
the following text appears in italics. Definitions, lemmas,
propositions, and corollaries are styled the same way.
\end{theorem}
%
% the environments 'definition', 'lemma', 'proposition', 'corollary',
% 'remark', and 'example' are defined in the LLNCS documentclass as well.
%
\begin{proof}
Proofs, examples, and remarks have the initial word in italics,
while the following text appears in normal font.
\end{proof}
For citations of references, we prefer the use of square brackets
and consecutive numbers. Citations using labels or the author/year
convention are also acceptable. The following bibliography provides
a sample reference list with entries for journal
articles~\cite{ref_article1}, an LNCS chapter~\cite{ref_lncs1}, a
book~\cite{ref_book1}, proceedings without editors~\cite{ref_proc1},
and a homepage~\cite{ref_url1}. Multiple citations are grouped
\cite{ref_article1,ref_lncs1,ref_book1},
\cite{ref_article1,ref_book1,ref_proc1,ref_url1}.

\begin{credits}
\subsubsection{\ackname} A bold run-in heading in small font size at the end of the paper is
used for general acknowledgments, for example: This study was funded
by X (grant number Y).

\subsubsection{\discintname}
It is now necessary to declare any competing interests or to specifically
state that the authors have no competing interests. Please place the
statement with a bold run-in heading in small font size beneath the
(optional) acknowledgments\footnote{If EquinOCS, our proceedings submission
system, is used, then the disclaimer can be provided directly in the system.},
for example: The authors have no competing interests to declare that are
relevant to the content of this article. Or: Author A has received research
grants from Company W. Author B has received a speaker honorarium from
Company X and owns stock in Company Y. Author C is a member of committee Z.
\end{credits}
%
% ---- Bibliography ----
%
% BibTeX users should specify bibliography style 'splncs04'.
% References will then be sorted and formatted in the correct style.
%
% \bibliographystyle{splncs04}
% \bibliography{mybibliography}
%
\begin{thebibliography}{8}
\bibitem{ref_article1}
Author, F.: Article title. Journal \textbf{2}(5), 99--110 (2016)

\bibitem{ref_lncs1}
Author, F., Author, S.: Title of a proceedings paper. In: Editor,
F., Editor, S. (eds.) CONFERENCE 2016, LNCS, vol. 9999, pp. 1--13.
Springer, Heidelberg (2016). \doi{10.10007/1234567890}

\bibitem{ref_book1}
Author, F., Author, S., Author, T.: Book title. 2nd edn. Publisher,
Location (1999)

\bibitem{ref_proc1}
Author, A.-B.: Contribution title. In: 9th International Proceedings
on Proceedings, pp. 1--2. Publisher, Location (2010)

\bibitem{ref_url1}
LNCS Homepage, \url{http://www.springer.com/lncs}, last accessed 2023/10/25
\end{thebibliography}
\end{document}
